\section{seasonal: R interface to
X-13ARIMA-SEATS}\label{seasonal-r-interface-to-x-13arima-seats}

\textbf{seasonal} is an easy-to-use R-interface to
\textbf{X-13ARIMA-SEATS}, a seasonal adjustment software
\textbf{produced, distributed, and maintained by the United States
Census Bureau}. X-13ARIMA-SEATS combines and extends the capabilities of
the older X-12ARIMA (developed by the Census Bureau) and the TRAMO-SEATS
(developed by the Bank of Spain) software packages.

If you are new to seasonal adjusmtent and X-13ARIMA-SEATS, you may use
the automated procedures to quickly produce seasonal adjustements of
some time series. The default settings in the core function generally do
a very good job. Start with the \emph{installation} and \emph{getting
started} section and skip the rest.

If you are familiar with seasonal adjusmtent and already know something
about X-13ARIMA-SEATS, you may benefit from the very close relationship
between the syntax in seasonal and X-13ARIMA-SEATS. Study the
\emph{X-13ARIMA-SEATS syntax} section and have a look at the
\href{https://github.com/christophsax/seasonal/wiki/Examples-of-X-13ARIMA-SEATS-in-R}{wiki},
where most examples from the original X-13ARIMA-SEATS manual are
reproduced in R. For more details on X-13ARIMA-SEATS, as well as for
explanations on the X-13ARIMA-SEATS syntax, see the
\href{http://www.census.gov/ts/x13as/docX13AS.pdf}{manual} or the
\href{http://www.census.gov/ts/x13as/pc/qrefX13ASpc.pdf}{quick
reference}.

\subsection{Installation}\label{installation}

To install directly from github to R, substitute your github
\texttt{'USERNAME'} and \texttt{'PASSWORD'}:

\begin{verbatim}
require(devtools)
install_github('seasonal', 'christophsax', auth_user = 'USERNAME', password = 'PASSWORD')
\end{verbatim}

seasonal includes the binary files of X-13ARIMA-SEATS. \textbf{No
separate download of the binaries is needed.}

\subsection{Getting started}\label{getting-started}

\texttt{seas} ist the core function of the seasonal package. By default,
\texttt{seas} calls the automatic procedures of X-13ARIMA-SEATS to
perform a seasonal adjustment that works very well in most
circumstances. It returns an object of class \texttt{seas} that contains
all necessary information on the adjustment process, as well as the
series. The \texttt{predict} method for \texttt{seas} objects returns
the adjusted series, the \texttt{plot} method shows a plot with the
unadjusted and the adjusted series.

\begin{verbatim}
 x <- seas(AirPassengers)
 predict(x)
 plot(x)
 
\end{verbatim}

The first argument must be a time series of class \texttt{ts}. By
default, \texttt{seas} calls the SEATS adjustment procedure. If you
prefer the X11 adjustment filter, use the following option (see the next
section for details on the syntax):

\begin{verbatim}
 seas(AirPassengers, x11 = list())
 
\end{verbatim}

Besides performing seasonal adjustment with SEATS, a default call of
\texttt{seas} invokes the following automatic procedures of
X-13ARIMA-SEATS: - ARIMA model search - Outlier detection - Detection of
trading day and Easter effects

Alternatively, all inputs may be entered manually, as in the following
example:

\begin{verbatim}
seas(AirPassengers,
     regression.variables = c("td1coef", "easter[1]", "ao1951.May"),
     arima.model = "(0 1 1)(0 1 1)",
     regression.aictest = NULL, outlier.types = "none"
)
\end{verbatim}

The \texttt{static} command reveals the static call from above that is
needed to replicate an automatic seasonal adjustment procedure:

\begin{verbatim}
static(x)
static(x, static.coeff = TRUE)  # also fixes the coefficients
\end{verbatim}

If you are using R Studio, the \texttt{inspect} command offers a way to
analyze and modify a seasonal adjustment procedure (see the section
below for details):

\begin{verbatim}
inspect(x)
\end{verbatim}

\subsection{X-13ARIMA-SEATS syntax}\label{x-13arima-seats-syntax}

Seasonal uses the same syntax as X-13ARIMA-SEATS. It is possible to
invoke most options that are available in X-13ARIMA-SEATS. For details
on the options, see the
\href{http://www.census.gov/ts/x13as/docX13AS.pdf}{manual}. The
X-13ARIMA-SEATS syntax uses \emph{specs} and \emph{arguments}, while
each spec may contain some arguments. \textbf{An additional
spec/argument can be added to the \texttt{seas} function by separating
spec and argument by a \texttt{.}.} For example, in order to set the
\texttt{variable} argument of the \texttt{regression} spec equal to
\texttt{td} and \texttt{ao1999.jan}, the input to \texttt{seas} looks
like this:

\begin{verbatim}
x <- seas(AirPassengers, regression.variable = c("td", "ao1965.jan"))
\end{verbatim}

Note that R vectors may be used as an input. If a \texttt{spec} is added
without any arguments, the \texttt{spec} should be set equal to an empty
\texttt{list()}. Several defaults of \texttt{seas} are such empty lists,
like the default \texttt{seats = list()}. See the help page
(\texttt{?seas}) for more details on the defaults.

It is possible to manipulate almost all inputs to X-13ARIMA-SEATS this
way. Most examples in the
\href{http://www.census.gov/ts/x13as/docX13AS.pdf}{manual} are
replicable in R. For instance, example 1 in section 7.1,

\begin{verbatim}
series { title  =  "Quarterly Grape Harvest" start = 1950.1
       period =  4
       data  = (8997 9401 ... 11346) }
arima { model = (0 1 1) }
estimate { }
\end{verbatim}

translates to R in the following way:

\begin{verbatim}
seas(AirPassengers,
     x11 = list(),
     arima.model = "(0 1 1)"
)
\end{verbatim}

\texttt{seas} takes care of the \texttt{series} spec, so no input beside
the time series has to be provided. As \texttt{seas} uses the SEATS
procedure by default, the use of X11 has to be specified manually. When
the \texttt{x11} spec is added as the input (as above), the mutually
exlusive and default \texttt{seats} spec is automatically disabled. With
\texttt{arima.model}, an additional spec/argument entry is added to the
input of X-13ARIMA-SEATS. As the spec cannot be used with the default
\texttt{automdl} spec, the latter is automatically disabled. The best
way to learn about the relationship between the syntax of
X-13ARIMA-SEATS and seasonal is to study the growing list of examples in
the
\href{https://github.com/christophsax/seasonal/wiki/Examples-of-X-13ARIMA-SEATS-in-R}{wiki}.

\subsubsection{Priority rules}\label{priority-rules}

There are several mutually exclusive specs in X-13ARIMA-SEATS. If more
than one mutually exclusive specs are included, X-13ARIMA-SEATS leads to
an error. In contrast, \texttt{seas} follows a set of priority rules,
where a lower priority is overwritten by a higher priority. Usually, the
default has the lowest priority, and is overwritten if one or several of
the following \texttt{spec} inputs are provided:

Model selection 1. \texttt{arima} 2. \texttt{pickmdl} 3.
\texttt{automdl} (default)

Adjustment procedure 1. \texttt{x11} 2. \texttt{seats} (default)

Regression procedure 1. \texttt{x11regression} 2. \texttt{regression}
(default)

\subsection{Output}\label{output}

\texttt{seas} returns an object of class \texttt{seas}, which is
basically a list with the following elements:

\begin{longtable}[c]{@{}ll@{}}
\hline\noalign{\medskip}
Element & Description
\\\noalign{\medskip}
\hline\noalign{\medskip}
\texttt{data} & An object of class \texttt{ts}, containing the
seasonally adjusted data, the raw data, the trend component, the
irregular component and the seasonal component. Accessing \texttt{data}
is for the advanced user. In general, the adjusted series should be
extracted with \texttt{predict}.
\\\noalign{\medskip}
\texttt{spc} & An object of class \texttt{spclist}, a list containing
everything that is send to X-13ARIMA-SEATS. Each \emph{spec} is on the
first level, each \emph{argument} is on the second level. Checking
\texttt{spc} is useful for debugging.
\\\noalign{\medskip}
\texttt{mdl} & A list with the model specification, similar to
\texttt{spc}. It typically contains \texttt{regression}, which contains
the regressors and parameter estimates, and \texttt{arima}, which
contains the ARIMA specification and the parameter estimates. The
\texttt{summary} function gives an overview of the model.
\\\noalign{\medskip}
\hline
\end{longtable}

\subsection{Graphs}\label{graphs}

\subsection{Inspect tool}\label{inspect-tool}

\subsection{License}\label{license}

When released, the R code in seasonal is licensed under GPL-3. The
package contains the X-13ARIMA-SEATS binary files from the United States
Census Bureau, which are in the public domain. According to the
\href{http://www.census.gov/ts/x13as/docX13AS.pdf}{manual} (page 1):

\begin{quote}
When it is released, the X-13ARIMA-SEATS program will be in the public
domain, and may be copied or transferred.
\end{quote}
