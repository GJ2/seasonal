\subsection{R interface to
X-13ARIMA-SEATS}\label{r-interface-to-x-13arima-seats}

\emph{seasonal} is an easy-to-use and (almost) full-featured R-interface
to X-13ARIMA-SEATS, the newest seasonal adjustment software developed by
the \href{http://www.census.gov/srd/www/x13as/}{United States Census
Bureau}. X-13ARIMA-SEATS combines and extends the capabilities of the
older X-12ARIMA (developed by the Census Bureau) and TRAMO-SEATS
(developed by the Bank of Spain).

If you are new to seasonal adjustment or X-13ARIMA-SEATS, the automated
procedures of \emph{seasonal} allow you to quickly produce good seasonal
adjustments of time series. Start with the
\hyperref[installation]{Installation} and
\hyperref[getting-started]{Getting started} section and skip the rest.

If you are familiar with X-13ARIMA-SEATS, you may benefit from the
flexible input and output structure of \emph{seasonal}. The package
allows you to use (almost) all commands of X-13ARIMA-SEATS, and it can
import (almost) all output generated by X-13ARIMA-SEATS. The only
exception is the `composite' spec, which is not supported. Read the
\hyperref[input]{Input} and \hyperref[output]{Output} sections and have
a look at the
\href{https://github.com/christophsax/seasonal/wiki/Examples-of-X-13ARIMA-SEATS-in-R}{wiki},
where the examples from the original X-13ARIMA-SEATS manual are
reproduced in R.

For more details on X-13ARIMA-SEATS, as well as for explanations on the
X-13ARIMA-SEATS syntax, see the
\href{http://www.census.gov/ts/x13as/docX13AS.pdf}{manual} or the
\href{http://www.census.gov/ts/x13as/pc/qrefX13ASpc.pdf}{quick
reference}.

\hyperdef{}{installation}{\subsubsection{Installation}\label{installation}}

To install the stable version directly from CRAN, type to the R console:

\begin{verbatim}
install.packages("seasonal")
\end{verbatim}

\emph{seasonal} does not includes the binary executables of
X-13ARIMA-SEATS. They need to be installed separately from
\href{http://www.census.gov/srd/www/x13as/x13down_pc.html}{here}
(Windows, filename \texttt{x13asall.zip}) or
\href{http://www.census.gov/srd/www/x13as/x13down_unix.html}{here}
(Linux, filename \texttt{x13asall.tar.gz}). My own compilation for Mac
OS-X can be obtained \href{mailto:christoph.sax@gmail.com}{upon
request}.

Download the file, unzip it and copy the folder to the desired location
in your file system. Next, you need to tell \emph{seasonal} where to
find the binary executables of X-13ARIMA-SEATS, by setting the specific
environmental variable \texttt{X13\_PATH}. This may be done during your
active session in R:

\begin{verbatim}
Sys.setenv(X13_PATH = "YOUR_X13_DIRECTORY")
\end{verbatim}

Exchange \texttt{YOUR\_X13\_DIRECTORY} with the path to your
installation of X-13ARIMA-SEATS. Note that the Windows path
\texttt{C:\textbackslash{}something\textbackslash{}somemore} has to be
entered UNIX-like \texttt{C:/something/somemore} or
\texttt{C:\textbackslash{}\textbackslash{}something\textbackslash{}\textbackslash{}somemore}.
You can always check your installation with:

\begin{verbatim}
checkX13()
\end{verbatim}

If you want to set the environmental variable permanently, you may do so
by adding it tho the \texttt{Renviron.site} file, which is located in
the \texttt{etc} subdirectory of your R home directory (use
\texttt{R.home()} in R to reveal the home directory).
\texttt{Renviron.site} does not exist by default; if not, you have to
create a file named \texttt{Renviron.site} with your favorite text
editor (on Windows, be careful with the `show extensions for known file
types' option, the extension \texttt{.site} may be hidden). Add the
following line to the file:

\begin{verbatim}
X13_PATH = YOUR_PATH_TO_X13
\end{verbatim}

Alternatively, use the system terminal (on Windows, it's called command
prompt; also, the \texttt{cd} command requires \texttt{\textbackslash{}}
instead of \texttt{/}):

\begin{verbatim}
cd YOUR_R_HOME_DIRECTORY/etc
echo X13_PATH = YOUR_PATH_TO_X13 >> Renviron.site
\end{verbatim}

There are other ways to set an environmental variable permanently in R,
see \texttt{?Startup}.

\hyperdef{}{getting-started}{\subsubsection{Getting
started}\label{getting-started}}

\texttt{seas} ist the core function of the \emph{seasonal} package. By
default, \texttt{seas} calls the automatic procedures of X-13ARIMA-SEATS
to perform a seasonal adjustment that works well in most circumstances.
It returns an object of class \texttt{"seas"} that contains all
necessary information on the adjustment process, as well as the series.
The \texttt{final} function returns the adjusted series, the
\texttt{plot} method shows a plot with the unadjusted and the adjusted
series.

\begin{verbatim}
m <- seas(AirPassengers)
final(x)
plot(x)
 
\end{verbatim}

The first argument of \texttt{seas} has to be a time series of class
\texttt{"ts"}. By default, \texttt{seas} calls the SEATS adjustment
procedure. If you prefer the X11 adjustment procedure (this is what
X-12ARIMA does), use the following option (see the the
\hyperref[input]{Input} section for more details about the syntax):

\begin{verbatim}
seas(AirPassengers, x11 = list())
 
\end{verbatim}

Besides performing seasonal adjustment with SEATS, a default call to
\texttt{seas} invokes the following automatic procedures of
X-13ARIMA-SEATS:

\begin{itemize}
\itemsep1pt\parskip0pt\parsep0pt
\item
  Transformation selection (log / no log)
\item
  Detection of trading day and Easter effects
\item
  Outlier detection
\item
  ARIMA model search
\end{itemize}

Alternatively, all inputs may be entered manually, as in the following
example:

\begin{verbatim}
seas(x = AirPassengers, regression.variables = c("td1coef", "easter[1]",
"ao1951.May"), arima.model = "(0 1 1)(0 1 1)", regression.aictest = NULL,
outlier = NULL, transform.function = "log")
\end{verbatim}

The \texttt{static} command reveals the static call from above that is
needed to replicate the automatic seasonal adjustment procedure:

\begin{verbatim}
static(m)
static(m, coef = TRUE)  # also fixes the coefficients
\end{verbatim}

If you are using RStudio, the \texttt{inspect} command offers a way to
analyze and modify a seasonal adjustment procedure (see the section
below for details):

\begin{verbatim}
inspect(m)
\end{verbatim}

\hyperdef{}{input}{\subsubsection{Input}\label{input}}

In \emph{seasonal}, it is possible to use almost the complete syntax of
X-13ARIMA-SEAT. This is done via the \texttt{...} argument in the
\texttt{seas} function. The X-13ARIMA-SEATS syntax uses \emph{specs} and
\emph{arguments}, with each spec optionally containing some arguments.
For details, see the official
\href{http://www.census.gov/ts/x13as/docX13AS.pdf}{manual}. These
spec-argument combinations can be added to \texttt{seas} by separating
the spec and the argument by a dot (\texttt{.}). For example, in order
to set the `variables' argument of the `regression' spec equal to
\texttt{td} and \texttt{ao1999.jan}, the input to \texttt{seas} looks
like this:

\begin{verbatim}
m <- seas(AirPassengers, regression.variables = c("td", "ao1955.jan"))
\end{verbatim}

Note that R vectors may be used as an input. If a spec is added without
any arguments, the spec should be set equal to an empty \texttt{list()}.
Several defaults of \texttt{seas} are empty lists, such as the default
\texttt{seats = list()}. See the help page (\texttt{?seas}) for more
details on the defaults.

It is possible to manipulate almost all inputs to X-13ARIMA-SEATS in
this way. For instance, example 1 in section 7.1 from the
\href{http://www.census.gov/ts/x13as/docX13AS.pdf}{manual},

\begin{verbatim}
series { title  =  "Quarterly Grape Harvest" start = 1950.1
       period =  4
       data  = (8997 9401 ... 11346) }
arima { model = (0 1 1) }
estimate { }
\end{verbatim}

translates to R in the following way:

\begin{verbatim}
seas(AirPassengers,
     x11 = list(),
     arima.model = "(0 1 1)"
)
\end{verbatim}

\texttt{seas} takes care of the `series' spec, and no input beside the
time series has to be provided. As \texttt{seas} uses the SEATS
procedure by default, the use of X11 has to be specified manually. When
the `x11' spec is added as an input (like above), the mutually exclusive
and default `seats' spec is automatically disabled. With
\texttt{arima.model}, an additional spec-argument is added to the input
of X-13ARIMA-SEATS. As the spec cannot be used in the same call as the
`automdl' spec, the latter is automatically disabled. The best way to
learn about the relationship between the syntax of X-13ARIMA-SEATS and
\emph{seasonal} is to study the comprehensive list of examples in the
\href{https://github.com/christophsax/seasonal/wiki/Examples-of-X-13ARIMA-SEATS-in-R}{wiki}.

There are several mutually exclusive specs in X-13ARIMA-SEATS. If more
than one mutually exclusive specs is included, a set of priority rule is
followed, where the lower priority is overwritten by the higher
priority:

\begin{itemize}
\itemsep1pt\parskip0pt\parsep0pt
\item
  Model selection

  \begin{enumerate}
  \def\labelenumi{\arabic{enumi}.}
  \itemsep1pt\parskip0pt\parsep0pt
  \item
    \texttt{arima}
  \item
    \texttt{pickmdl}
  \item
    \texttt{automdl} (default)
  \end{enumerate}
\item
  Adjustment procedure

  \begin{enumerate}
  \def\labelenumi{\arabic{enumi}.}
  \itemsep1pt\parskip0pt\parsep0pt
  \item
    \texttt{x11}
  \item
    \texttt{seats} (default)
  \end{enumerate}
\end{itemize}

\hyperdef{}{output}{\subsubsection{Output}\label{output}}

\emph{seasonal} has a very flexible mechanism to read data from
X-13ARIMA-SEATS. With the \texttt{series} function, it is possible to
import almost all output that can be generated by X-13ARIMA-SEATS. For
example, the following command shows the forcasts of the ARIMA model:

\begin{verbatim}
m <- seas(AirPassengers)
series(m, "forecast.forecasts")
\end{verbatim}

Because the \texttt{forecast.save = "forecasts"} argument has not been
specified in the model call, \texttt{series} re-evaluates the call with
the `forecast' spec enabled. It is also possible to show more than one
ouput table at the same time:

\begin{verbatim}
series(m, c("forecast.forecasts", "d1"))
\end{verbatim}

You can use either the unique short names of X-13 (such as \texttt{d1}),
or the the long names (such as \texttt{forecasts}). Because the long
table names are not unique, they need to be combined with the spec name
(\texttt{forecast}). See \texttt{?series} for a complete list of
options.

Note that re-evaluation doubles the overall computation time. If you
want to speed it up, you have to be explicit about the output in the
model call:

\begin{verbatim}
m <- seas(AirPassengers, forecast.save = "forecasts")
series(m, "forecast.forecasts")
\end{verbatim}

Some specs, like `slidingspans' and `history', are time consuming.
Re-evaluation allows you to separate these specs from the basic model
call:

\begin{verbatim}
m <- seas(AirPassengers)
series(m, "history.saestimates")
series(m, "slidingspans.sfspans")
\end{verbatim}

The \texttt{out} function shows the full content of the \texttt{.out}
text file from X-13ARIMA-SEATS in a console viewer. It can also be
searched:

\begin{verbatim}
out(m)
out(m, search = "regARIMA model residuals")
\end{verbatim}

\subsubsection{Graphs}\label{graphs}

There are several graphical tools to analyze a \texttt{seas} model. The
main plot function draws the seasonally adjusted and unadjusted series,
as well as the outliers. Optionally, it also draws the trend of the
seasonal decomposition:

\begin{verbatim}
m <- seas(AirPassengers, regression.aictest = c("td", "easter"))
plot(m)
plot(m, outliers = FALSE)
plot(m, trend = TRUE)
\end{verbatim}

The \texttt{monthplot} function allows for a monthwise plot (or
quarterwise, with the same function name) of the seasonal and the SI
component:

\begin{verbatim}
monthplot(m)
monthplot(m, choice = "irregular")
\end{verbatim}

Also, many standard R function can be used to analyze a \texttt{"seas"}
model:

\begin{verbatim}
pacf(resid(m))
spectrum(diff(resid(m)))
plot(density(resid(m)))
qqnorm(resid(m))
\end{verbatim}

\subsubsection{Inspect tool}\label{inspect-tool}

The \texttt{inspect} function is a powerful graphical tool for choosing
a seasonal adjustment model. It uses the \texttt{manipulate} package and
can only be used with the (free)
\href{http://www.rstudio.com/ide/}{RStudio IDE}. The goal of
\texttt{inspect} is to summarize all relevant options, plots and
statistics that should be usually considered. \texttt{inspect} uses a
\texttt{"seas"} object as its only argument:

\begin{verbatim}
inspect(m)
\end{verbatim}

The \texttt{inspect} function opens an interactive window that allows
for the manipulation of a number of arguments. It offers several views
to analyze the series graphically. With each change, the adjustment
process and the visualizations are recalculated. Summary statics are
shown in the R console.

With the `Show static call' option, a replicable static call is also
shown in the console. Note that this option will double the time for
recalculation, as the static function also tests the static call each
time.

\subsubsection{License}\label{license}

\emph{seasonal} is free and open source, licensed under GPL-3. It has
been developed for the use at the Swiss State Secretariat of Economic
Affairs and is not connected to the development of X-13ARIMA-SEATS,
which is in the Public Domain.

This is a new package, and it may still contain bugs. Please report them
on \href{https://github.com/christophsax/seasonal}{Github} or send me an
\href{mailto:christoph.sax@gmail.com}{e-mail}. Thank you!
